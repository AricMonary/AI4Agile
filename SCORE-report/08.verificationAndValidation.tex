\section{Verification and Validation}

\subsection{UI/UX Feedback}

The application is primarily composed of individual web panels integrated into Jira’s existing User Interface. As a result, each web panel was able to be created, tested, and reviewed independent from Jira Cloud. The isolation was used to conduct input and output testing but the modules were generally tested within the context of Jira to ensure that the user experience is seamless between our integrated app and the existing Jira Cloud platform.

\subsection{Testing}

Testing was performed on a component and integration level. Component testing was separated into two categories: text processing and relationship visualization.

Text processing component level testing was based upon results from three categories of inputs: ideal, disjoint, and semi-ideal. Ideal inputs were those where all information belonged to clear categories. Disjoint inputs were those where all information was completely separate each with separate categories. Semi-ideal inputs had both outlier pieces of information and information that distinctly can be clustered with other info. Disjoint inputs were used to evaluate edge case usages of the app. To represent the average case, semi-ideal inputs were used since natural language understanding from a vector perspective is ambiguous without context.

At an integration level, all possible user paths were explored as each text process could be done independently or sequentially. For example, a user can complete the entire process by entering by decomposing an epic, optimizing the generated stories, and then generate tasks from the optimized stories. However, a user may choose to only optimize stories that were manually entered and then go on to generate tasks from the optimized stories. 
