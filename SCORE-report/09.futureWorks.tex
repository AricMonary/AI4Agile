\section{Future Works and Conclusion}
\label{conclusion}
As part of an academic capstone project, regardless of whether the team would advance into the next stage with the SCORE competition, we plan to continue on with the project for the next few months. The team has so far defined a vision with loose deliverables in mind. For this next phase, we will deploy the project to the Atlassian Marketplace, hence also collecting feedback from potential actual customers, which will be helpful for further improvements.  At the meantime, the team have noticed several possible improvements for our main features. For instance, in the epic decomposition process, we might implement dynamic clustering of stories, that automatically adjust the numbers of stories generated, instead of relying on user provided parameters. The main improvement for story optimization would be in optimizing the algorithm to be faster, as the current wait time is relatively long compared to our goal, as it caused a noticeable delay. 

We are also looking into switching from the current supporting package in Java, to the Stanza Natural Language Processing (SNLP) \cite{NLP1} package in Python. The SNLP package has a word dependency feature that could improve the results of task generation, especially in generating complete sentences rather than fragments. The relationship visualization component also has multiple adjustments that could be made, such as filtering and ease-of-use controls for zooming and panning. 

In conclusion, the team has completed the tasks within the scope as we defined based upon the AI4Agile project proposal. In this paper, we presented the team's structure and SE process, as well as all major stakeholders invovled. We then described in details the lifecycle throughout the project, covering requirements, design, implementation and final user story based testing and demonstration. The team have plans to continue the project in the near future as part of our capstone project, and will look into explore several directions where improvements on the main features could be done. The ultimate goal of the team is to deploy the project to be adopted by potential real clients. 