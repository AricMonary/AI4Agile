\section{Future Works}
Now that this project is beyond the initial competition’s timeline and restraints, the team can explore a new scope as the focus shifts towards viewing the project more heavily through the lens of being an academic capstone, without the prompt as the primary guide. The team has so far defined a vision with loose deliverables in mind for this next time period, with the intent that these things be better defined upon return from the winter development break and with the finding of real customers. In further detail below are the plans for this, including the specific deliverables.

During the second phase of development, the project scope will be based upon the successful deployment of the app to the Atlassian Marketplace, along with input from an actual customer.  As a result of the app becoming customer-oriented, the specific features of the application will be dependent on the guidance of the customer as elicited at the beginning of the second phase. Despite the dependence on customer inputs, general guidelines have been predetermined for areas or features that would most benefit from improvement or creation. For epic decomposition, that improvement is in the form of aiming for dynamic clustering of stories, as opposed to the current clustering with a slider for the user to adjust if the amount of stories doesn’t fit well with the size of the given epic. The main improvement for story optimization would be in optimizing the algorithm to yield results faster, as the current wait time is high relative to the team’s goal times. The final AI component improvement would be for task generation, in the form of switching from the current supporting package in Java to the Stanza Natural Language Processing package in Python. This switch would aid in improving the results from task generation, especially in lessening the outputting of sentence fragments instead of complete sentences, due to a word dependency feature within the Python package. The switch comes as a result of research presented in “A Novel System for Generating Simple Sentences from Complex and Compound Sentences.” These AI improvements are centered on the user experience, as long wait times or results that need multiple adjustments to be used can decrease the time benefits of using the application’s refinement support over the manual method. The relationship visualization component has a variety of ways it could be improved, in ways such as providing more relationship options to be shown, filtering options to cover those choices so they don’t all need to be displayed at once, and ease-of-use controls for zooming and panning in the form of on-screen buttons instead of implicit mouse controls. All of these deliverables are potential plans for the upcoming project period, but their approval or further details will depend heavily on the results of the customer focus shift.