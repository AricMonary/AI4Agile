\section{ Preliminary Understanding}
What follows is a discussion of the background of this project, as well as the team’s interpretations of the undertaking from the start. This project originated from the SCORE 2021 competition prompt titled “AI4Agile: Developing AI-enabled tool support for user story refinement in JIRA”. This prompt was sponsored by professor Hoa Khanh Dam, and tied into previous research he was involved with, one of the papers of which was linked directly within\cite{b1}. The prompt further went into the reasoning for AI usage in this setting and the vision for such an application, which along the way this team has expanded upon and further detailed as the project progressed in order to better understand the vision.

The prompt given for this project listed four primary requirements towards the goal of aiding in agile user story refinement. These four were:

\begin{enumerate}
	\item Recommend user stories that are decomposed from an epic.
	\item Recommend smaller user stories are [sic] split from bigger user stories.
	\item Recommend subtasks derived from a user story.
	\item A visualization (e.g. a graph) of epics, user stories and tasks, and their relationships.
\end{enumerate}

In addition to these requirements, it was specified that the application was to be integrated into the platform of Atlassian’s Jira software. While there are different versions of Jira available to develop for, Jira Cloud was chosen over Jira Server, since the team viewed it as being more relevant to future trends in the workplace due to its focus not being on hosting on-premises for work environments. An abridged version of the vision for this project as it connects to the scheme of artificial intelligence (AI) or machine learning usage for software development was also given. A more complete vision was detailed in the linked research paper, as it contained proposals and persuasions towards a grander project for AI use in agile assistance.  

The grander project’s intent focused primarily on risk management via AI, while this competition project’s scope, which was defined from analysis of the project prompt and input from an early on defined customer, had a primary focus of providing a time savings for users refining user stories. The risk management project involved the user of historical data and subsequently an increased utilization of AI to process that data and provide more personalized feedback and tools to teams. In that context, the necessity of AI and the form it would take was clearer, while in the prompt its role was left more open to interpretation. In the end, the team determined that the artificial intelligence component’s purpose would be centered on processing text into increasingly smaller portions in order to support user story refinement.

While the project’s goal on a functional level was described as supporting user story refinement, the motivation behind that given in the prompt involved reducing reliance of the process on team members’ experience levels, and supporting that process so that fewer differences are introduced by people refining via their own methods. This also ties back to the ideal of the parent project for risk management, as it would improve the consistency of the refinement process, since the overlap or contradiction of rules for refining backlog items when done by hand is likely to cause inconsistency in resolution methods used, making for less predictable outcomes and backlog item sizes.
