\section{Introduction}
AI4Agile is an application for Jira Cloud, made to assist software development teams who use agile methods in user story streamlining. The application consists of three processes to refine user stories from an epic into tasks. An additional visualization component, integrated into Jira, offers suggestions to users and visualizes both explicit and implicit relationships amongst epics, user stories, and tasks.

The primary three processing can be encompassed as epic decomposition, story optimization, and task generation. Epic decomposition consists of taking a templatized epic inputted by the user and clustering requirements together into user stories based upon subject. Story optimization further breaks down potentially large stories into smaller stories. Certain stories may not be subject to change depending on their initial size. Lastly, the user stories will be broken into the tasks based on parts-of-speech in the stories description. At each process, the results will be offered as suggestions to the user, allowing them to select which user stories and tasks they’d like to add to their Jira board. The user can further change any suggestions along with regenerating new suggestions based upon new parameters. The input for each process will be based on the user’s selected suggestions and inputted stories. Epics, user stories, and tasks may be selected and a graph field will populate showing a selectable explicit and implicit relationships which may be rendered in a tree or cluster form.

The report is composed of ten sections. The second section outlines our initial understanding of the application of AI to the agile development process and our vision of the app . The third section covers the structure of the team and methods of collaborating with stakeholders. The fourth section summarizes the decomposition of requirements into functional and non-functional categories. Section five outlines the design and section six explains the implementation along with the differences between it and the initial design. Section seven goes through two use cases that encapsulate the primary functionalities that were implemented. Section eight describes the methods for evaluating the user interface and experience in addition to testing the main processes. Section nine explores the future prospects of the project. Lastly, section 10 offers reflection on the outcome of the project.

